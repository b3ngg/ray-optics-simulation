\chapter{Einleitung}
Die physikalischen Grundlagen für die Reflexion von Lichtstrahlen 
ist für das menschliche Gehirn einfach zu verstehen. 
So kann der Mensch die Bahn von reflektierten Lichtwellen 
an einfachen geometrischen Formen fast intuitiv vorhersagen. 
In einem komplexeren Umfeld mit einer erhöhten Anzahl an 
physikalisch relevanten Entitäten wird die Prognose 
der Gesamtheit aller stattfindenden Einwirkungen auf den 
Lichtweg jedoch für Menschen zunehmend schwierig 
zu prognostizieren und vor allem zeitaufwendig.
Wenn auch der initiale Aufwand für die digitale Implementierung dieser physikalischen 
optischen Grundlagen der Reflexion größer sein mag, 
ist dies vor allem in erwähnten umfangreicheren Situationen praktikabel und ermöglicht 
schnelle Iterationen bezüglich der Veränderung 
des Resultats im Zusammenhang mit der Ausgangssituation. 

Die vorliegende Arbeit beschäftigt sich mit der Frage, 
wie eine solche digitale Simulation umgesetzt werden kann. 
Hierbei wird nicht nur auf die Umsetzung der tatsächlichen 
Reflexion eingegangen, sondern auch auf 
den Aufbau einer solchen Computer-Simulation, 
sodass eine Erweiterung hinsichtlich weiterer strahlenoptischer 
Phänomene möglich ist. \\ 
Zur Vereinfachung wird von einer idealisierten, zweidimensionalen, 
evakuierten Umgebung ausgegangen, 
in der Lichtstrahlen nicht an Intensität verlieren
und die Oberflächen von Hindernissen störungsfrei sind.
\newpage
Die Quellenlage bezüglich der hier verwendeten Grundlagen der geometrischen Optik ist als sehr sicher zu bewerten. 
Der Aufbau und die technische Umsetzung der Simulation entstammen fast ausschließlich aus eigenen Überlegungen.