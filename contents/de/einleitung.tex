\chapter{Einleitung}
Die physikalischen Grundlagen der Reflexion von Lichtstrahlen folgen simpler Logik. 
So lassen sich die Bahnen von reflektierten Lichtwellen an einfachen 
geometrischen Formen nahezu intuitiv prognostizieren.
In einem komplexeren Umfeld mit einer erhöhten Anzahl an 
physikalisch relevanten Entitäten wird die Prognose 
der Gesamtheit aller stattfindenden Einwirkungen auf den 
Lichtweg jedoch für Menschen zunehmend schwierig 
zu prognostizieren und vor allem zeitaufwendig.
Wenn auch der initiale Aufwand für die digitale Implementierung dieser physikalischen 
optischen Grundlagen der Reflexion größer sein mag, 
ist dies vor allem in erwähnten umfangreicheren Situationen praktikabel und ermöglicht 
schnelle Iterationen bezüglich der Veränderung 
des Resultats im Zusammenhang mit der Ausgangssituation. 

Die vorliegende Arbeit beschäftigt sich mit der Frage, 
wie eine solche digitale Simulation umgesetzt werden kann. 
Hierbei wird nicht nur auf die Umsetzung der tatsächlichen 
Reflexion eingegangen, sondern auch auf 
den Aufbau einer solchen Computer-Simulation, 
sodass eine Erweiterung hinsichtlich weiterer strahlenoptischer 
Phänomene möglich ist. \\ 
Zur Vereinfachung wird von einer idealisierten, zweidimensionalen, 
evakuierten Umgebung ausgegangen, 
in der Lichtstrahlen nicht an Intensität verlieren
und die Oberflächen von Hindernissen störungsfrei sind.
\newpage
Die Quellenlage bezüglich der hier verwendeten physikalischen Grundlagen der geometrischen Optik ist als sehr sicher zu bewerten. 
Der Aufbau und die technische Umsetzung der Simulation entstammen jedoch fast ausschließlich aus Überlegungen des Autors.
Etwaige Fehler hat allein der Autor zu verantworten.