\chapter{Aufbau}
\section{Lichtstrahl}
In der geometrischen Optik geht man bei einem Lichtstrahl \code{Ray} 
von einem Strahl aus, der sich von einem Ursprung $ A $  
geradlinig in eine Richtung mit dem Winkel 
$ \alpha $ ausbreitet. \parencite[vgl.][S. 1041]{tipler2015physik} mathematisch 
wird dieser als Halbgerade angesehen $ \overrightarrow{AB} $. 
Da der Punkt $ B $ für die Simulation keinen Nutzen hat, ist es 
sinnvoll den Startpunkt als Ortsvektor $ \vec{O} $ \code{origin} innerhalb der Welt
und den Winkel \code{angle}, in den der Strahl ausgeht direkt zu speichern. 

\section{Lichtweg}
Es muss davon ausgegangen werden, dass ein Lichtstrahl in der 
Simulation durch die Kollision mit anderen Objekten seine Richtung verändert. 
Ab einer Richtungsänderung wird der bisherige Lichtstrahl als Lichtweg \code{Line} 
gespeichert, um in später zu visualisieren.
Hierbei besteht ein Lichtstrahl aus den Ortsvektoren $ \vec{O} $ \code{origin} dem 
Ursprung des Lichtstrahls und $ \vec{E} $ \code{end} dem Punkt der Kollision.