\chapter{Fazit}
Es ist faszinierend, wie sich in der geometrischen Optik aus simpelsten Regeln schnell spannendes und komplexes Verhalten entwickeln kann.
Damit zu experimentieren wäre ohne die Assistenz einer Computer-Simulation nur schwer möglich. 

Die aktuelle Version wird mir eine große Hilfe sein, wenn ich in meiner Facharbeit 
das komplexe Verhalten von Lichtstrahlen im Bezug auf die Chaos-Theorie behandeln werde.
Dennoch ist diese Simulation noch weit davon entfernt, vollständig zu sein. 
Dank der Separierung von Hindernis und Material ist eine Erweiterung im Bezug auf die Brechung von Licht und weiteren Phänomenen einfach möglich.
Aber auch die aktuelle Version kann noch optimiert werden. So ist die Performance bei vielen Lichtstrahlen und/oder Hindernissen nicht sonderlich gut.
Hier gilt es in Zukunft vor allem das Testen auf Schnittpunkte zu optimieren. 